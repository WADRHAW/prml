\subsubsection{Conditional Gaussian distribution}
(preliminaries)
\begin{conclusion}
\begin{equation}
\begin{bmatrix}A&B\\C&D\\\end{bmatrix}^{-1}=\begin{bmatrix}M&-MBD^{-1}\\-D^{-1}CM&D^{-1}+D^{-1}CMBD^{-1}\\\end{bmatrix}
\label{eqn:schurcomplement}
\end{equation}
where $A,B,D,D$ are respectively
$p\times{}p,p\times{}q,q\times{}p,q\times{}q$ matrices and
$M=(A-BD^{-1}C)^{-1}$. Note that $D$ is required to be
\emph{invertible}. $M$ is known as the \emph{Schur complement} of the
matrix on the left-hand side of \eqref{eqn:schurcomplement} w.r.t. the
submatrix $D$.
\end{conclusion}
\begin{proof}
Multiply
$\begin{bmatrix}M&-MBD^{-1}\\-D^{-1}CM&D^{-1}+D^{-1}CMBD^{-1}\\\end{bmatrix}$
by $\begin{bmatrix}A&B\\C&D\\\end{bmatrix}$ and check whether it
results
$I_{(p+q)\times{}(p+q)}=\begin{bmatrix}I_{p\times{}p}&0\\0&I_{q\times{}q}\\\end{bmatrix}$.


Firstly, $MA-MBD^{-1}C=M(A-BD^{-1}C)=I_{p\times{}p}$.


Secondly, $MB-MBD^{-1}D=MB-MB=0$.


Thirdly,
    $-D^{-1}CMA+D^{-1}C+D^{-1}CMBD^{-1}C=D^{-1}C(-MA+I+MBD^{-1}C)=D^{-1}CM(-A+M^{-1}+BD^{-1}C)=0$.


Finally, $-D^{-1}CMB+D^{-1}D+D^{-1}CMBD^{-1}D=I_{q\times{}q}$.
\end{proof}


Suppose $\mathbf{x}$ is a $D$-dimensional vector with Gaussian
distribution $\mathcal{N}(\mathbf{x}\vert{}\boldsymbol{\mu},\Sigma)$
and we partition $\mathbf{x}$ into two disjoint subsets $\mathbf{x}_{a}$
and $\mathbf{x}_{b}$. We can always put $\mathbf{x}_{a}$ on the first
$M$ components of $\mathbf{x}$ by \emph{permutation} over $\boldsymbol{\mu}$
and $\Sigma$ without loss of generality.
\begin{itemize}
\item $\mathbf{x}=\begin{bmatrix}\mathbf{x}_{a}\\\mathbf{x}_{b}\\\end{bmatrix}$
\item
$\boldsymbol{\mu}=\begin{bmatrix}\boldsymbol{\mu}_{a}\\\boldsymbol{\mu}_{b}\\\end{bmatrix}$
\item
$\Sigma=\begin{bmatrix}\Sigma_{aa}&\Sigma_{ab}\\\Sigma_{ba}&\Sigma_{bb}\\\end{bmatrix}$
and the following properties result from $\Sigma^{\mathrm{T}}=\Sigma$
    \begin{itemize}
    \item $\Sigma_{aa}=\Sigma_{aa}^{\mathrm{T}}$ and $\Sigma_{bb}=\Sigma_{bb}^{\mathrm{T}}$
    \item $\Sigma_{ab}^{\mathrm{T}}=\Sigma_{ba}$
    \end{itemize}
\item $\Lambda=\Sigma^{-1}=\begin{bmatrix}\Lambda_{aa}&\Lambda_{ab}\\\Lambda_{ba}&\Lambda_{bb}\\\end{bmatrix}$
\end{itemize}


Using this partitioning we firstly obtain:
\begin{equation}
\begin{split}
&-\frac{1}{2}(\mathbf{x}-\boldsymbol{\mu})^{\mathrm{T}}\Sigma^{-1}(\mathbf{x}-\boldsymbol{\mu})\\
=&-\frac{1}{2}\begin{bmatrix}(\mathbf{x}_{a}-\boldsymbol{\mu}_{a})^{\mathrm{T}}&(\mathbf{x}_{b}-\boldsymbol{\mu}_{b})^{\mathrm{T}}\\\end{bmatrix}\begin{bmatrix}\Lambda_{aa}&\Lambda_{ab}\\\Lambda_{ba}&\Lambda_{bb}\\\end{bmatrix}\begin{bmatrix}\mathbf{x}_{a}-\boldsymbol{\mu}_{a}\\\mathbf{x}_{b}-\boldsymbol{\mu}_{b}\end{bmatrix}\\
=&-\frac{1}{2}\begin{bmatrix}(\mathbf{x}_{a}-\boldsymbol{\mu}_{a})^{\mathrm{T}}\Lambda_{aa}+(\mathbf{x}_{b}-\boldsymbol{\mu}_{b})^{\mathrm{T}}\Lambda_{ba}&(\mathbf{x}_{a}-\boldsymbol{\mu}_{a})^{\mathrm{T}}\Lambda_{ab}+(\mathbf{x}_{b}-\boldsymbol{\mu}_{b})^{\mathrm{T}}\Lambda_{bb}\\\end{bmatrix}\begin{bmatrix}\mathbf{x}_{a}-\boldsymbol{\mu}_{a}\\\mathbf{x}_{b}-\boldsymbol{\mu}_{b}\end{bmatrix}\\
=&-\frac{1}{2}\{(\mathbf{x}_{a}-\boldsymbol{\mu}_{a})^{\mathrm{T}}\Lambda_{aa}(\mathbf{x}_{a}-\boldsymbol{\mu}_{a})+(\mathbf{x}_{b}-\boldsymbol{\mu}_{b})^{\mathrm{T}}\Lambda_{ba}(\mathbf{x}_{a}-\boldsymbol{\mu}_{a})+(\mathbf{x}_{a}-\boldsymbol{\mu}_{a})^{\mathrm{T}}\Lambda_{ab}(\mathbf{x}_{b}-\boldsymbol{\mu}_{b})+(\mathbf{x}_{b}-\boldsymbol{\mu}_{b})^{\mathrm{T}}\Lambda_{bb}(\mathbf{x}_{b}-\boldsymbol{\mu}_{b})\}
\end{split}    
\label{eqn:abquadratic}
\end{equation}
We find that as a function of $\mathbf{x}_{a}$, this is again a
quadratic form.
\begin{conclusion}
If two sets of random variables are jointly Gaussian, then the
conditional probability distribution of one set conditioned on the
other is agian Gaussian.
\end{conclusion}


To determine the mean and covariance of
$\Pr(\mathbf{x}_{a}\vert\mathbf{x}_{b})$, we should firstly note that
the exponent in a general
$\mathcal{N}(\mathbf{x}\vert\boldsymbol{\mu}\,\Sigma)$ can be written
as:
\begin{equation}
-\frac{1}{2}(\mathbf{x}-\boldsymbol{\mu})^{\mathrm{T}}\Sigma^{-1}(\mathbf{x}-\boldsymbol{\mu})=-\frac{1}{2}\mathbf{x}^{\mathrm{T}}\Sigma^{-1}\mathbf{x}+\mathbf{x}^{\mathrm{T}}\Sigma^{-1}\boldsymbol{\mu}-\frac{1}{2}\boldsymbol{\mu}^{\mathrm{T}}\Sigma^{-1}\boldsymbol{\mu}
\label{eqn:expandexponent}
\end{equation}


We pick out all terms that are second order in $\mathbf{x}_{a}$ from
\eqref{eqn:abquadratic}:
$-\frac{1}{2}\mathbf{x}_{a}^{\mathrm{T}}\Lambda_{aa}\mathbf{x}_{a}$.
Thus the covariance of $\Pr(\mathbf{x}_{a}\vert\mathbf{x}_{b})$ is
$\Lambda_{aa}^{-1}$.


We pick out all terms that are linear in $\mathbf{x}_{a}$ from
\eqref{eqn:abquadratic}:
\begin{equation}
\begin{split}
&\frac{1}{2}\mathbf{x}_{a}^{\mathrm{T}}\Lambda_{aa}\boldsymbol{\mu}_{a}+\frac{1}{2}\boldsymbol{\mu}_{a}^{\mathrm{T}}\Lambda_{aa}\mathbf{x}_{a}-\frac{1}{2}\mathbf{x}_{a}^{\mathrm{T}}\Lambda_{ab}(\mathbf{x}_{b}-\boldsymbol{\mu}_{b})-\frac{1}{2}(\mathbf{x}_{b}-\boldsymbol{\mu}_{b})^{\mathrm{T}}\Lambda_{ba}\mathbf{x}_{a}\\
=&\mathbf{x}_{a}^{\mathrm{T}}\Lambda_{aa}\boldsymbol{\mu}_{a}-\mathbf{x}_{a}^{\mathrm{T}}\Lambda_{ab}(\mathbf{x}_{b}-\boldsymbol{\mu}_{b})\quad{}(\because{}\mathbf{x}^{\mathrm{T}}A\mathbf{y}=\mathbf{y}^{\mathrm{T}}A^{\mathrm{T}}\mathbf{x}\text{
        and }\Lambda_{ba}^{\mathrm{T}}=\Lambda_{ab})\\
=&\mathbf{x}_{a}^{\mathrm{T}}\{\Lambda_{aa}\boldsymbol{\mu}_{a}-\Lambda_{ab}(\mathbf{x}_{b}-\boldsymbol{\mu}_{b})\}
\end{split}
\label{eqn:linearinxa}
\end{equation}
Thus the mean of $\Pr(\mathbf{x}_{a}\vert\mathbf{x}_b)$ is
$\Lambda_{aa}^{-1}\{\Lambda_{aa}\boldsymbol{\mu}_{a}-\Lambda_{ab}(\mathbf{x}_{b}-\boldsymbol{\mu}_{b})\}=\boldsymbol{\mu}_{a}-\Lambda_{aa}^{-1}\Lambda_{ab}(\mathbf{x}_{b}-\boldsymbol{\mu}_{b})$.


If we can ensure $\Sigma_{bb}$ to be invertible (how?), 
\putansline{6}{5}
we can make use of \eqref{eqn:schurcomplement} to express
$\Lambda_{aa}$ as
$(\Sigma_{aa}-\Sigma_{ab}\Sigma_{bb}^{-1}\Sigma_{ba})^{-1}$
and hence:
\begin{equation}
\Sigma_{a\vert{}b}=\Lambda_{aa}^{-1}=(\Sigma_{aa}-\Sigma_{ab}\Sigma_{bb}^{-1}\Sigma_{ba})
\label{eqn:covagivenb}    
\end{equation}
we can also make use of \eqref{eqn:schurcomplement} to express
$\Lambda_{ab}$ as
$-\Lambda_{aa}\Sigma_{ab}\Sigma_{bb}^{-1}$
and hence:
\begin{equation}
\begin{split}
\mu_{a\vert{}b}&=\boldsymbol{\mu}_{a}-\Lambda_{aa}^{-1}\Lambda_{ab}(\mathbf{x}_{b}-\boldsymbol{\mu}_{b})\\
&=\boldsymbol{\mu}_{a}+\Lambda_{aa}^{-1}\Lambda_{aa}\Sigma_{ab}\Sigma_{bb}^{-1}(\mathbf{x}_{b}-\boldsymbol{\mu}_{b})\\
&=\boldsymbol{\mu}_{a}+\Sigma_{ab}\Sigma_{bb}^{-1}(\mathbf{x}_{b}-\boldsymbol{\mu}_{b})
\end{split}
\label{eqn:meanagivenb}
\end{equation}
