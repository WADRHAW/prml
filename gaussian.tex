\subsection{The Gaussian Distribution}
Preliminaries:
\begin{conclusion}
For quadratic form $\mathbf{x}^{\mathrm{T}}A\mathbf{x}$, matrix $A$ can be taken to be symmetric without loss of generality.
\end{conclusion}
\begin{proof}
$\because\mathbf{x}^{\mathrm{T}}A\mathbf{x}=\sum_{i=1}^{D}\sum_{j=1}^{D}x_{i}x_{j}A_{ij}\quad\therefore\forall\text{ subscript pair }(i,j)\text{ its contribution
is }x_{i}x_{j}(A_{ij}+A_{ji})$\\
Thus, for any asymmetric matrix $B$, it is equivalent to $A$ where
$A_{ij}=A_{ji}=\frac{B_{ij}+B_{ji}}{2}$
\end{proof}


\begin{conclusion}
Matrix $A$ is symmetric $\Leftrightarrow$
$A\mathbf{x}\cdot\mathbf{y}=\mathbf{x}\cdot{}A\mathbf{y}$.
\end{conclusion}
\begin{proof}
($\Rightarrow$)
    $\because{}A\mathbf{x}\cdot{}\mathbf{y}=(A\mathbf{x})^{\mathrm{T}}\mathbf{y}=\mathbf{x}^{\mathrm{T}}A\mathbf{y}=\mathbf{x}\cdot{}A\mathbf{y}$\\
($\Leftarrow$) $A\mathbf{e}_{i}\cdot{}\mathbf{e}_{j}=A_{ji}=\mathbf{e}_{i}\cdot{}A\mathbf{e}_{j}=A_{ij}$\\
$\because{}\forall{}(i,j):A_{ij}=A_{ji}\quad{}\therefore{}A=A^{\mathrm{T}}$
\end{proof}


\begin{conclusion}
Real symmetric matrix $A_{n\times{}n}$ has $n$ real eigenvalues (counting
        multiplicities).
\end{conclusion}
\begin{proof}
Suppose $A\mathbf{x}=\lambda{}\mathbf{x}$ where
$\mathbf{x}\in{}\mathbb{C}^{n}$. Let
$q=\bar{\mathbf{x}}^{\mathrm{T}}A\mathbf{x}$\\
$\because\bar{q}=\mathbf{x}^{\mathrm{T}}A\bar{\mathbf{x}}=\bar{q}^{\mathrm{T}}=\bar{\mathbf{x}}^{\mathrm{T}}A\mathbf{x}=q\quad{}\therefore{}q$
is actually real.\\
$\because{}q=\bar{\mathbf{x}}^{\mathrm{T}}A\mathbf{x}=\bar{\mathbf{x}}^{\mathrm{T}}\lambda{}\mathbf{x}=\lambda{}\Vert{}\mathbf{x}\Vert{}^{2}\quad{}\therefore{}\lambda{}$
is also real.
\end{proof}


\begin{conclusion}
Symmetric matrix $A_{n\times{}n}$ has $n$ eigenvectors forming an orthonormal set.
\end{conclusion}
\begin{proof}
(\emph{Assumption}) for $i=1,\ldots,n-1\quad{}(n>1)$, symmetric matrix $A_{i\times{}i}$
has $i$ orthonormal eigenvectors and $A_{n\times{}n}$ is symmetric but
doesn't have $n$ orthonormal eigenvectors.

Pick $\lambda_1$, one of $A_{n\times{}n}$'s $n$ eigenvalues and pick a
corresponding unit eigenvector $\mathbf{u}_1$ (i.e.,
        $A\mathbf{u}_{1}=\lambda_{1}\mathbf{u}_{1}$). Let
$W=\text{Nul }\mathbf{u}_{1}^{\mathrm{T}}$ be the
null space of $\mathbf{u}_1$. Obviously, it is a $n-1$ dimensional
subspace of $\mathbb{R}^{n}$. Then we could choose the basis for $W$
consisting of $n-1$ orthonormal vectors
$\mathbb{B}=\{\mathbf{u}_{2}\ldots,\mathbf{u}_{n}\}$ and let
$P_{\mathbb{B}}=[\mathbf{u}_{2},\ldots,\mathbf{u}_{n}]$. Although
$P_{\mathbb{B}}$ is not square, its rank is $n-1$ and thus
gives an one-to-one and onto mapping from $\mathbb{R}^{n-1}$ to $W$ 
which implies the existence of its \emph{left inverse} $P_{\mathbb{B}}^{-1}$.
$$
\begin{array}{ccc}
W=\text{Nul }\mathbf{u_1}^{\mathrm{\textnormal{T}}} & &\Re^{n-1}\\
\mathbf{w} & \autorightleftharpoons{$P_{\mathcal{B}}^{-1}$}{$P_{\mathcal{B}}$}
&[\mathbf{w}]_{\mathcal{B}}\\
\end{array}
$$


Then we prove that multiplication by $A$ defines a linear
transformation $T:W\rightarrow{}W$:\\
$\because{}A$ is
symmetric.$\quad{}\therefore{}\forall{}\mathbf{w}\in{}W:\mathbf{u}_{1}\cdot{}A\mathbf{w}=A\mathbf{u}_{1}\cdot{}\mathbf{w}=\lambda_{1}\mathbf{u}_{1}\cdot{}\mathbf{w}=0$\\
Note that the codomain of mapping $T$ is $W$ but the range may not.
$$
\begin{array}{ccc}
W=\{\mathbf{w}\in\Re^{n}\vert\mathbf{u_1}^{\mathrm{\textnormal{T}}}\mathbf{w}=0\}
& \autorightarrow{$T:A$}{} & V=\{A\mathbf{w}\vert\mathbf{w}\in
W\}\subseteq W\\
& &\\
\downarrow\quad P_{\mathcal{B}}^{-1} & & \downarrow\quad P_{\mathcal{B}}^{-1}\\
& &\\
\Re^{n-1}=\{[\mathbf{w}]_{\mathcal{B}}\vert\mathbf{w}\in W\} &
\autorightarrow{$T^{'}:M$}{} &
M[\mathbf{w}]_{\mathcal{B}}=[A\mathbf{w}]_{\mathcal{B}}\in\Re^{n-1}\\
%\Re^{n-1}=\{[A\mathbf{w}]_{\mathcal{B}}\vert A\mathbf{w}\in V\}\\
\end{array}
$$


Now our task is to prove the existence of matrix $M$ in the above
relationships:\\
Suppose $\mathbf{w}=r_{2}\mathbf{u}_{2}+\ldots+r_{n}\mathbf{u}_{n}$
and since $T$ is a linear transformation which preserves addition and
scalar multiplication, we have:
\begin{equation}
T(\mathbf{w})=r_{2}T(\mathbf{u}_2)+\ldots+r_{n}T(\mathbf{u}_{n})
\label{eqn:lineart}
\end{equation}
Rewrite \eqref{eqn:lineart} in basis $\mathbb{B}$ we have:
\begin{equation}
[T(\mathbf{w})]_{\mathbb{B}}=r_{2}[T(\mathbf{u}_2)]_{\mathbb{B}}+\ldots+r_{n}[T(\mathbf{u}_{n})]_{\mathbb{B}}
\end{equation}
Note that
$[\mathbf{w}]_{\mathbb{B}}=[r_{2}\quad{}\cdots{}\quad{}r_{n}]^{\mathrm{T}}$
and thus:
\begin{equation}
[T(\mathbf{w})]_{\mathbb{B}}=M[\mathbf{w}]_{\mathbb{B}}
\label{eqn:tandm}
\end{equation}
where
$M=[[T(\mathbf{u}_2)]_{\mathbb{B}}\quad{}\cdots{}\quad{}[T(\mathbf{u}_{n})]_{\mathbb{B}}]$


$M$ is related to $A$ in terms of eigenvalues and eigenvectors:\\
Suppose
$M[\mathbf{x}]_{\mathbb{B}}=\lambda{}[\mathbf{x}]_{\mathbb{B}}$,
    according to \eqref{eqn:tandm},
    $[A\mathbf{x}]_{\mathbb{B}}=\lambda{}[\mathbf{x}]_{\mathbb{B}}$.
    $\because{}[\cdot]_{\mathbb{B}}$ is actually multiplication by
    $P_{\mathbb{B}}^{-1}:W\rightarrow{}\mathbb{R}^{n-1}$ which is
    a linear transformation with one-to-one and onto properties.
    $\therefore{}\mathbf{x}$ and $[\mathbf{x}]_{\mathbb{B}}$ possess
    uniquely correspondence and $\mathbf{x}$ is eigenvector of $A$.
    Besides, since we choose the inverse mapping $P_{\mathbb{B}}$ to
    be constructed by basis of $W$, $\mathbf{x}\in{}W$.


If there is an orthonormal basis of $\mathbb{R}^{n-1}$ consisting of
$n-1$ eigenvectors of
$M:[\mathbf{x}_2]_{\mathbb{B}},\ldots,[\mathbf{x}_{n}]_{\mathbb{B}}$,
    then $\mathbf{x}_{2},\ldots,\mathbf{x}_{n}$ are eigenvectors of
    $A$.
    $\because{}\forall{}i=2,\ldots,n:\mathbf{x}_{i}\in{}W\quad{}\therefore{}\mathbf{u}_{1}\cdot{}\mathbf{x}_{i}=0$.
    Now the remaining task is to prove
    $\mathbf{x}_{2},\ldots,\mathbf{x}_{n}$ are orthonormal. Remember that
    $[\mathbf{x}_{2}]_{\mathbb{B}},\ldots,[\mathbf{x}_{n}]_{\mathbb{B}}$
    are orthonormal so that if multiplication by $P_{\mathbb{B}}$
    preserves dot products (in turn length), then their
    correspondences are also orthonormal. So the next steps are:
\begin{itemize}
\item multiplication by $P_{\mathbb{B}}$ preserves dot products.
\item $M$ has $n-1$ orthonormal eigenvectors.
\end{itemize}


Preserving dot products means
$\forall{}\mathbf{x},\mathbf{y}\in{}\mathbb{R}^{n-1}:\mathbf{x}\cdot{}\mathbf{y}=(P_{\mathbb{B}}\mathbf{x})\cdot{}(P_{\mathbb{B}}\mathbf{y})$.
It is straightforward by noticing that
$P_{\mathbb{B}}^{\mathrm{T}}P_{\mathbb{B}}=I_{(n-1)\times{}(n-1)}$.
Besides, equality is symmetric and we can say $[\cdot]_{\mathbb{B}}$
preserves dot products (i.e., both $P_{\mathbb{B}}$ and $P_{\mathbb{B}}^{-1}$).


To show that $M$ has $n-1$ orthonormal eigenvectors seems difficult,
   however, note that by induction, a symmetric matrix
   $M_{(n-1)\times{}(n-1)}$ has $n-1$ orthonormal eigenvectors. Thus,
   we try to prove $M$ is symmetric.\\
$\because{}A$ is symmetric.
$\therefore{}A\mathbf{x}\cdot{}\mathbf{y}=\mathbf{x}\cdot{}A\mathbf{y}$ 
$\because{}[]_{\mathbb{B}}$ is an isomorphism and preserves dot
products. $\therefore{}$ according to \eqref{eqn:tandm},
    $M[\mathbf{x}]_{\mathbb{B}}\cdot{}[\mathbf{y}]_{\mathbb{B}}=[\mathbf{x}]_{\mathbb{B}}\cdot{}M[\mathbf{y}]_{\mathbb{B}}$
    which implies that $M$ is symmetric.


Up to now, we showed that $A$ has $n$ orthonormal eigenvectors
$\{\mathbf{u}_{1},\mathbf{x}_{2},\ldots,\mathbf{x}_{n}\}$ which
contradicts our initial assumption. Thus, there doesn't exist such a
$n$ and $\forall{}n:$ symmetric matrix $A_{n\times{}n}$ has $n$ orthonormal eigenvectors.
\end{proof}


\begin{conclusion}
Symmetric matrix $A$ is orthonormally diagonalizable, i.e.,
          $A=U\Lambda{}U^{\mathrm{T}}$ where $U$ is orthogonal
          (i.e., columns are unit orthogonal vectors and rows are unit
           orthogonal vectors) and $\Lambda$ is diagonal matrix.
\end{conclusion}
\begin{proof}
We have proved that symmetric matrix $A_{n\times{}n}$ has $n$
orthonormal eigenvectors
$\{\mathbf{u}_{1},\ldots,\mathbf{u}_{n}\}\quad{}(A\mathbf{u}_{i}=\lambda_{i}\mathbf{u}_{i})$.
Let $U=[\mathbf{u}_{1}\quad{}\cdots{}\quad{}\mathbf{u}_{n}]$ and
$\Lambda{}=\begin{bmatrix}\lambda_1& & \\ &\ddots&\\ &
&\lambda_n\\\end{bmatrix}$, we show that $A=U\Lambda{}U^{\mathrm{T}}$:
$\because{}U$ consists of unit orthonormal columns. $\therefore{}$ right multiplied by
$U,AU=[\lambda_{1}\mathbf{u}_{1}\quad{}\cdots{}\lambda_{n}\mathbf{u}_{n}]$
and $U\Lambda{}U^{\mathrm{T}}U=U\Lambda{}=[\lambda_{1}\mathbf{u}_{1}\quad{}\cdots{}\lambda_{n}\mathbf{u}_{n}]$
\end{proof}


The multivariate Gaussian distribution takes the form:
\begin{definition}
\begin{equation}
\mathcal{N}(\mathbf{x}\vert\boldsymbol{\mu{}},\Sigma)=\frac{1}{(2\pi)^{\frac{D}{2}}}\frac{1}{\vert\Sigma\vert^{\frac{1}{2}}}\exp{}\{-\frac{1}{2}(\mathbf{x}-\boldsymbol{\mu})^{\mathrm{T}}\Sigma^{-1}(\mathbf{x}-\boldsymbol{\mu})\}
\label{eqn:defgaussian}
\end{equation}
\end{definition}
where $\boldsymbol{\mu{}}$ is the mean vector, $\Sigma$ is a $D\times{}D$
covariance matrix with its determinant $\vert{}\Sigma\vert$.


The functional dependence of the Gaussian on $\mathbf{x}$ is through
the quadratic form:
\begin{equation}
\Delta^2=(\mathbf{x}-\boldsymbol{\mu})^{\mathrm{T}}\Sigma^{-1}(\mathbf{x}-\boldsymbol{\mu})
\label{eqn:mahalanobis}
\end{equation}
The quantity $\Delta$ is called the \emph{Mahalanobis distance} from
$\boldsymbol{\mu}$ to $\mathbf{x}$.


(Based on above propositions) We can choose $\Sigma^{-1}$ to be
symmetric without loss of generality. Actually, we firstly choose the
covariance matrix $\Sigma$ to be symmetric, then $\Sigma$ is
orthogonal diagonalizable:
\begin{equation}
\begin{split}
\Sigma{}&=U\Lambda{}U^{\mathrm{T}}\\
&=\begin{bmatrix}\mathbf{u}_{1}&\cdots{}&\mathbf{u}_{D}\\\end{bmatrix}\begin{bmatrix}\lambda_1&
& \\ &\ddots&\\ &
&\lambda_{D}\\\end{bmatrix}\begin{bmatrix}\mathbf{u}_{1}^{\mathrm{T}}\\\vdots{}\\\mathbf{u}_{D}^{\mathrm{T}}\\\end{bmatrix}\\
&=\begin{bmatrix}\lambda_{1}\mathbf{u}_{1}&\cdots{}&\lambda_{D}\mathbf{u}_{D}\\\end{bmatrix}\begin{bmatrix}\mathbf{u}_{1}^{\mathrm{T}}\\\vdots{}\\\mathbf{u}_{D}^{\mathrm{T}}\\\end{bmatrix}\\
&=\sum_{i=1}^{D}\lambda_{i}\mathbf{u}_{i}\mathbf{u}_{i}^{\mathrm{T}}
\end{split}
\label{eqn:factcovariance}
\end{equation}
Because $\Lambda$ is diagonal matrix, its inverse is $\Lambda^{-1}=\begin{bmatrix}\frac{1}{\lambda_{1}}& & \\ &\ddots{}& \\
        & &\frac{1}{\lambda_{D}}\\\end{bmatrix}$. 
In addition, $U$ is orthogonal ($UU^{\mathrm{T}}=I$). Thus, we have:
\begin{equation}
\Sigma^{-1}=\sum_{i=1}^{D}\frac{1}{\lambda_{i}}\mathbf{u}_{i}\mathbf{u}_{i}^{\mathrm{T}}
\label{eqn:factprecise}
\end{equation}
This equation also implicitly proves that
\begin{conclusion}
Inverse of symmetric matrix is also symmetric.
\end{conclusion}


Substitute \eqref{eqn:factprecise} into \eqref{eqn:mahalanobis}, the
quadratic form becomes:
\begin{equation}
\Delta^{2}=\sum_{i=1}^{D}\frac{y_{i}^2}{\lambda_{i}}
\end{equation}
where
$y_{i}=\mathbf{u}_{i}^{\mathrm{T}}(\mathbf{x}-\boldsymbol{\mu})$. We
can interpret $\{y_{i}\}$ as a new coordinate system by projecting
$(\mathbf{x}-\boldsymbol{\mu})$ to $\{\mathbf{u}_{i}\}$. The
functional dependence of
$\mathbf{y}=\begin{bmatrix}y_{1}&\cdot{}&y_{D}\\\end{bmatrix}$ on $\mathbf{x}$ is captured by
the following equation:
\begin{equation}
\mathbf{y}=U^{\mathrm{T}}(\mathbf{x}-\boldsymbol{\mu})
\label{eqn:replacevariable}    
\end{equation}


To be properly normalized, $\Sigma$ should be \emph{positive
    definite}. Otherwise, see ``Chap12'':
\putansline{6}{4}


Integral by substitution ($\mathbf{x}$ replaced by $\mathbf{y}$)
    requires calculating \emph{Jacobi determinant} where
    $J_{ij}=\frac{\partial{}x_{i}}{\partial{}y_{j}}=$. From
    \eqref{eqn:replacevariable} we know that
    $\mathbf{x}=U\mathbf{y}+\boldsymbol{\mu}$ and thus
    $J_{ij}=\frac{\partial{}x_{i}}{\partial{}y_{j}}=U_{ji}^{\mathrm{T}}$.
    Obviously, this implies that $J=U$. Then we have:
\begin{equation}
\begin{split}
\vert{}J\vert{}^{2}&=\vert{}U\vert{}^{2}\\
&=\vert{}U^{\mathrm{T}}\vert{}\vert{}U\vert{}\quad{}(\because{}\vert{}A\vert{}=\vert{}A^{\mathrm{T}}\vert{})\\
&=\vert{}U^{\mathrm{T}}U\vert{}\quad{}(\because{}\vert{}A\vert{}\vert{}B\vert{}=\vert{}AB\vert{})\\
&=\vert{}I\vert{}=1
\end{split}
\end{equation}
Use these properties of determinant in above derivation, we also have:
\begin{equation}
\begin{split}
\vert{}\Sigma\vert{}&=\vert{}U\Lambda{}U^{\mathrm{T}}\vert{}\\
&=\vert{}U\vert{}\vert{}\Lambda\vert{}\vert{}U^{\mathrm{T}}\vert{}\\
&=\vert{}U^{\mathrm{T}}U\vert{}\vert{}\Lambda\vert{}\\
&=\prod_{i=1}^{D}\lambda_{i}
\end{split}
\label{eqn:detofsymmetric}
\end{equation}
Up to now, we know the substitution results:
\begin{equation}
\begin{split}
\Pr(\mathbf{y})&=\Pr(\mathbf{x})\vert{}J\vert{}\\
&=\frac{1}{(2\pi)^{\frac{D}{2}}}\frac{1}{\prod_{i=1}^{D}\lambda_{i}^{\frac{1}{2}}}\exp{}\{-\frac{1}{2}\sum_{i=1}^{D}\frac{y_{i}^2}{\lambda_i}\}\\
&=\prod_{i=1}^{D}\frac{1}{(2\pi\lambda_i)^{\frac{1}{2}}}\exp{}\{-\frac{y_{i}^2}{2\lambda_i}\}
\end{split}
\label{eqn:subgaussian}
\end{equation}
From the perspective of $\mathbf{y}$ coordinate system, it is
straightforward to show that:
\begin{equation}
\int{}\Pr(\mathbf{y})\text{d}\mathbf{y}=\prod_{i=1}^{D}\int_{-\infty}^{\infty}\frac{1}{(2\pi\lambda_{i})^{\frac{1}{2}}}\exp{}\{-\frac{y_{i}^2}{2\lambda_{i}}\}\text{d}y_{i}=1
\end{equation}


We now check the moment and second moment of Gaussian. By substitution
$\mathbf{z}=\mathbf{x}-\boldsymbol{\mu}$ we have
$\mathbb{E}[\mathbf{x}]=\frac{1}{(2\pi)^{\frac{D}{2}}}\frac{1}{\vert\Sigma\vert^{\frac{1}{2}}}\int{}\exp{}\{-\frac{1}{2}\mathbf{z}^{\mathrm{T}}\Sigma^{-1}\mathbf{z}\}(\mathbf{z}+\boldsymbol{\mu})\text{d}\mathbf{z}$.
Note that $\exp(\cdot)$ is an even function and the integral region is
the whole $\mathbb{R}^{D}$. Thus the term of exponential function
multiplied by $z$ will vanish and
$\mathbb{E}[\mathbf{x}]=\boldsymbol{\mu}$.


Univariate case we consider $\mathbb{E}[x^2]$, while in the multivariate
case ($D$ dimensional) we have $D^2$ pairs $x_{i}x_{j}$. Thus we
consider:
\begin{equation}
\begin{split}
\mathbb{E}[\mathbf{x}\mathbf{x}^{\mathrm{T}}]&=\frac{1}{(2\pi)^{\frac{D}{2}}}\frac{1}{\vert\Sigma\vert^{\frac{1}{2}}}\int{}\exp{}\{-\frac{1}{2}(\mathbf{x}-\boldsymbol{\mu})^{\mathrm{T}}\Sigma^{-1}(\mathbf{x}-\boldsymbol{\mu})\}\mathbf{x}\mathbf{x}^{\mathrm{T}}\text{d}\mathbf{x}\\
&=\frac{1}{(2\pi)^{\frac{D}{2}}}\frac{1}{\vert\Sigma\vert^{\frac{1}{2}}}\int{}\exp{}\{-\frac{1}{2}\mathbf{z}^{\mathrm{T}}\Sigma^{-1}\mathbf{z}\}(\mathbf{z}+\boldsymbol{\mu})(\mathbf{z}+\boldsymbol{\mu})^{\mathrm{T}}\text{d}\mathbf{z}
\end{split}        
\label{eqn:secondmomentgaussian}
\end{equation}
Consider
$(\mathbf{z}+\boldsymbol{\mu})(\mathbf{z}+\boldsymbol{\mu})^{\mathrm{T}}$
we firstly find that $\mathbf{z}\boldsymbol{\mu}^{\mathrm{T}}$ and
$\boldsymbol{\mu}\mathbf{z}^{\mathrm{T}}$ will vanish because of the
same reason leveraged in computing moment of Gaussian above. Secondly,
     $\boldsymbol{\mu}\boldsymbol{\mu}^{\mathrm{T}}$ doesn't contain
     integral variable $\mathbf{z}$ and thus contributes
     $\boldsymbol{\mu}\boldsymbol{\mu}^{\mathrm{T}}$ to final integral
     result since $\Pr(\cdot)$ is normalized. Remember that from
     \eqref{eqn:replacevariable} we have $\mathbf{z}=U\mathbf{y}$ and
     substitution results:
\begin{equation}
\begin{split}
&\frac{1}{(2\pi)^{\frac{D}{2}}}\frac{1}{\vert{}\Sigma\vert^{\frac{1}{2}}}\int{}\exp{}\{-\frac{1}{2}\mathbf{z}^{\mathrm{T}}\Sigma^{-1}\mathbf{z}\}\mathbf{z}\mathbf{z}^{\mathrm{T}}\text{d}\mathbf{z}\\
=&\frac{1}{(2\pi)^{\frac{D}{2}}}\frac{1}{\vert{}\Sigma\vert^{\frac{1}{2}}}\int{}\exp{}\{-\sum_{i=1}^{D}\frac{y_{i}^2}{2\lambda_{i}}\}(U\mathbf{y})(U\mathbf{y})^{\mathrm{T}}\text{d}\mathbf{y}\\
=&\frac{1}{(2\pi)^{\frac{D}{2}}}\frac{1}{\vert{}\Sigma\vert^{\frac{1}{2}}}\int{}\exp{}\{-\sum_{i=1}^{D}\frac{y_{i}^2}{2\lambda_{i}}\}(\sum_{j=1}^{D}y_{j}\mathbf{u}_{j})(\sum_{k=1}^{D}y_{k}\mathbf{u}_{k}^{\mathrm{T}})\text{d}\mathbf{y}\\
=&\frac{1}{(2\pi)^{\frac{D}{2}}}\frac{1}{\vert{}\Sigma\vert^{\frac{1}{2}}}\sum_{j=1}^{D}\sum_{k=1}^{D}\mathbf{u}_{j}\mathbf{u}_{k}^{\mathrm{T}}\int{}\exp{}\{-\sum_{i=1}^{D}\frac{y_{i}^2}{2\lambda_{i}}\}y_{j}y_{k}\text{d}\mathbf{y}\\
\end{split}
\label{eqn:mprocsecmom}
\end{equation}
Since $y_{j}y_{k}$ is odd function of $y_{j},y_{k}$ respectively, the
term with different $j,k$ will vanish. Besides,
     $\int{}\mathcal{N}(x\vert{}\mu,\sigma)x^{2}\text{d}x=\sigma$.
     Thus, we can further simplify \eqref{eqn:mprocsecmom} to:
\begin{equation}
\frac{1}{(2\pi)^{\frac{D}{2}}}\frac{1}{\vert{}\Sigma\vert^{\frac{1}{2}}}\sum_{j=1}^{D}\mathbf{u}_{j}\mathbf{u}_{j}^{\mathrm{T}}\int{}\exp{}\{-\sum_{i=1}^{D}\frac{y_{i}^2}{2\lambda_{i}}\}y_{j}^{2}\text{d}\mathbf{y}=\sum_{j=1}^{D}\mathbf{u}_{j}\mathbf{u}_{j}^{\mathrm{T}}\lambda_{j}=\Sigma
\label{eqn:gaussiansecmm}
\end{equation}
Based on above analysis and calculation we have:
\begin{equation}
\begin{split}
\text{cov}[\mathbf{x}]&=\mathbb{E}[(\mathbf{x}-\mathbb{E}[\mathbf{x}])(\mathbf{x}-\mathbb{E}[\mathbf{x}])^{\mathrm{T}}]\\
&=\mathbb{E}[\mathbf{x}\mathbf{x}^{\mathrm{T}}]-\mathbb{E}[\mathbf{x}]\mathbb{E}[\mathbf{x}]^{\mathrm{T}}\\
&=\Sigma+\boldsymbol{\mu}\boldsymbol{\mu}^{\mathrm{T}}-\boldsymbol{\mu}\boldsymbol{\mu}^{\mathrm{T}}\\
&=\Sigma
\end{split}
\label{eqn:gaussiancov}
\end{equation}
Because $\Sigma$ governs the covariance, it is called \emph{covariance
    matrix} and $\Sigma^{-1}$ is called the \emph{precision matrix}.

\subsubsection{Conditional Gaussian distribution}
(preliminaries)
\begin{conclusion}
\begin{equation}
\begin{bmatrix}A&B\\C&D\\\end{bmatrix}^{-1}=\begin{bmatrix}M&-MBD^{-1}\\-D^{-1}CM&D^{-1}+D^{-1}CMBD^{-1}\\\end{bmatrix}
\label{eqn:schurcomplement}
\end{equation}
where $A,B,D,D$ are respectively
$p\times{}p,p\times{}q,q\times{}p,q\times{}q$ matrices and
$M=(A-BD^{-1}C)^{-1}$. Note that $D$ is required to be
\emph{invertible}. $M$ is known as the \emph{Schur complement} of the
matrix on the left-hand side of \eqref{eqn:schurcomplement} w.r.t. the
submatrix $D$.
\end{conclusion}
\begin{proof}
Multiply
$\begin{bmatrix}M&-MBD^{-1}\\-D^{-1}CM&D^{-1}+D^{-1}CMBD^{-1}\\\end{bmatrix}$
by $\begin{bmatrix}A&B\\C&D\\\end{bmatrix}$ and check whether it
results
$I_{(p+q)\times{}(p+q)}=\begin{bmatrix}I_{p\times{}p}&0\\0&I_{q\times{}q}\\\end{bmatrix}$.


Firstly, $MA-MBD^{-1}C=M(A-BD^{-1}C)=I_{p\times{}p}$.


Secondly, $MB-MBD^{-1}D=MB-MB=0$.


Thirdly,
    $-D^{-1}CMA+D^{-1}C+D^{-1}CMBD^{-1}C=D^{-1}C(-MA+I+MBD^{-1}C)=D^{-1}CM(-A+M^{-1}+BD^{-1}C)=0$.


Finally, $-D^{-1}CMB+D^{-1}D+D^{-1}CMBD^{-1}D=I_{q\times{}q}$.
\end{proof}


Suppose $\mathbf{x}$ is a $D$-dimensional vector with Gaussian
distribution $\mathcal{N}(\mathbf{x}\vert{}\boldsymbol{\mu},\Sigma)$
and we partition $\mathbf{x}$ into two disjoint subsets $\mathbf{x}_{a}$
and $\mathbf{x}_{b}$. We can always put $\mathbf{x}_{a}$ on the first
$M$ components of $\mathbf{x}$ by \emph{permutation} over $\boldsymbol{\mu}$
and $\Sigma$ without loss of generality.
\begin{itemize}
\item $\mathbf{x}=\begin{bmatrix}\mathbf{x}_{a}\\\mathbf{x}_{b}\\\end{bmatrix}$
\item
$\boldsymbol{\mu}=\begin{bmatrix}\boldsymbol{\mu}_{a}\\\boldsymbol{\mu}_{b}\\\end{bmatrix}$
\item
$\Sigma=\begin{bmatrix}\Sigma_{aa}&\Sigma_{ab}\\\Sigma_{ba}&\Sigma_{bb}\\\end{bmatrix}$
and the following properties result from $\Sigma^{\mathrm{T}}=\Sigma$
    \begin{itemize}
    \item $\Sigma_{aa}=\Sigma_{aa}^{\mathrm{T}}$ and $\Sigma_{bb}=\Sigma_{bb}^{\mathrm{T}}$
    \item $\Sigma_{ab}^{\mathrm{T}}=\Sigma_{ba}$
    \end{itemize}
\item $\Lambda=\Sigma^{-1}=\begin{bmatrix}\Lambda_{aa}&\Lambda_{ab}\\\Lambda_{ba}&\Lambda_{bb}\\\end{bmatrix}$
\end{itemize}


Using this partitioning we firstly obtain:
\begin{equation}
\begin{split}
&-\frac{1}{2}(\mathbf{x}-\boldsymbol{\mu})^{\mathrm{T}}\Sigma^{-1}(\mathbf{x}-\boldsymbol{\mu})\\
=&-\frac{1}{2}\begin{bmatrix}(\mathbf{x}_{a}-\boldsymbol{\mu}_{a})^{\mathrm{T}}&(\mathbf{x}_{b}-\boldsymbol{\mu}_{b})^{\mathrm{T}}\\\end{bmatrix}\begin{bmatrix}\Lambda_{aa}&\Lambda_{ab}\\\Lambda_{ba}&\Lambda_{bb}\\\end{bmatrix}\begin{bmatrix}\mathbf{x}_{a}-\boldsymbol{\mu}_{a}\\\mathbf{x}_{b}-\boldsymbol{\mu}_{b}\end{bmatrix}\\
=&-\frac{1}{2}\begin{bmatrix}(\mathbf{x}_{a}-\boldsymbol{\mu}_{a})^{\mathrm{T}}\Lambda_{aa}+(\mathbf{x}_{b}-\boldsymbol{\mu}_{b})^{\mathrm{T}}\Lambda_{ba}&(\mathbf{x}_{a}-\boldsymbol{\mu}_{a})^{\mathrm{T}}\Lambda_{ab}+(\mathbf{x}_{b}-\boldsymbol{\mu}_{b})^{\mathrm{T}}\Lambda_{bb}\\\end{bmatrix}\begin{bmatrix}\mathbf{x}_{a}-\boldsymbol{\mu}_{a}\\\mathbf{x}_{b}-\boldsymbol{\mu}_{b}\end{bmatrix}\\
=&-\frac{1}{2}\{(\mathbf{x}_{a}-\boldsymbol{\mu}_{a})^{\mathrm{T}}\Lambda_{aa}(\mathbf{x}_{a}-\boldsymbol{\mu}_{a})+(\mathbf{x}_{b}-\boldsymbol{\mu}_{b})^{\mathrm{T}}\Lambda_{ba}(\mathbf{x}_{a}-\boldsymbol{\mu}_{a})+(\mathbf{x}_{a}-\boldsymbol{\mu}_{a})^{\mathrm{T}}\Lambda_{ab}(\mathbf{x}_{b}-\boldsymbol{\mu}_{b})+(\mathbf{x}_{b}-\boldsymbol{\mu}_{b})^{\mathrm{T}}\Lambda_{bb}(\mathbf{x}_{b}-\boldsymbol{\mu}_{b})\}
\end{split}    
\label{eqn:abquadratic}
\end{equation}
We find that as a function of $\mathbf{x}_{a}$, this is again a
quadratic form.
\begin{conclusion}
If two sets of random variables are jointly Gaussian, then the
conditional probability distribution of one set conditioned on the
other is agian Gaussian.
\end{conclusion}


To determine the mean and covariance of
$\Pr(\mathbf{x}_{a}\vert\mathbf{x}_{b})$, we should firstly note that
the exponent in a general
$\mathcal{N}(\mathbf{x}\vert\boldsymbol{\mu}\,\Sigma)$ can be written
as:
\begin{equation}
-\frac{1}{2}(\mathbf{x}-\boldsymbol{\mu})^{\mathrm{T}}\Sigma^{-1}(\mathbf{x}-\boldsymbol{\mu})=-\frac{1}{2}\mathbf{x}^{\mathrm{T}}\Sigma^{-1}\mathbf{x}+\mathbf{x}^{\mathrm{T}}\Sigma^{-1}\boldsymbol{\mu}-\frac{1}{2}\boldsymbol{\mu}^{\mathrm{T}}\Sigma^{-1}\boldsymbol{\mu}
\label{eqn:expandexponent}
\end{equation}


We pick out all terms that are second order in $\mathbf{x}_{a}$ from
\eqref{eqn:abquadratic}:
$-\frac{1}{2}\mathbf{x}_{a}^{\mathrm{T}}\Lambda_{aa}\mathbf{x}_{a}$.
Thus the covariance of $\Pr(\mathbf{x}_{a}\vert\mathbf{x}_{b})$ is
$\Lambda_{aa}^{-1}$.


We pick out all terms that are linear in $\mathbf{x}_{a}$ from
\eqref{eqn:abquadratic}:
\begin{equation}
\begin{split}
&\frac{1}{2}\mathbf{x}_{a}^{\mathrm{T}}\Lambda_{aa}\boldsymbol{\mu}_{a}+\frac{1}{2}\boldsymbol{\mu}_{a}^{\mathrm{T}}\Lambda_{aa}\mathbf{x}_{a}-\frac{1}{2}\mathbf{x}_{a}^{\mathrm{T}}\Lambda_{ab}(\mathbf{x}_{b}-\boldsymbol{\mu}_{b})-\frac{1}{2}(\mathbf{x}_{b}-\boldsymbol{\mu}_{b})^{\mathrm{T}}\Lambda_{ba}\mathbf{x}_{a}\\
=&\mathbf{x}_{a}^{\mathrm{T}}\Lambda_{aa}\boldsymbol{\mu}_{a}-\mathbf{x}_{a}^{\mathrm{T}}\Lambda_{ab}(\mathbf{x}_{b}-\boldsymbol{\mu}_{b})\quad{}(\because{}\mathbf{x}^{\mathrm{T}}A\mathbf{y}=\mathbf{y}^{\mathrm{T}}A^{\mathrm{T}}\mathbf{x}\text{
        and }\Lambda_{ba}^{\mathrm{T}}=\Lambda_{ab})\\
=&\mathbf{x}_{a}^{\mathrm{T}}\{\Lambda_{aa}\boldsymbol{\mu}_{a}-\Lambda_{ab}(\mathbf{x}_{b}-\boldsymbol{\mu}_{b})\}
\end{split}
\label{eqn:linearinxa}
\end{equation}
Thus the mean of $\Pr(\mathbf{x}_{a}\vert\mathbf{x}_b)$ is
$\Lambda_{aa}^{-1}\{\Lambda_{aa}\boldsymbol{\mu}_{a}-\Lambda_{ab}(\mathbf{x}_{b}-\boldsymbol{\mu}_{b})\}=\boldsymbol{\mu}_{a}-\Lambda_{aa}^{-1}\Lambda_{ab}(\mathbf{x}_{b}-\boldsymbol{\mu}_{b})$.


If we can ensure $\Sigma_{bb}$ to be invertible (how?), 
\putansline{6}{5}
we can make use of \eqref{eqn:schurcomplement} to express
$\Lambda_{aa}$ as
$(\Sigma_{aa}-\Sigma_{ab}\Sigma_{bb}^{-1}\Sigma_{ba})^{-1}$
and hence:
\begin{equation}
\Sigma_{a\vert{}b}=\Lambda_{aa}^{-1}=(\Sigma_{aa}-\Sigma_{ab}\Sigma_{bb}^{-1}\Sigma_{ba})
\label{eqn:covagivenb}    
\end{equation}
we can also make use of \eqref{eqn:schurcomplement} to express
$\Lambda_{ab}$ as
$-\Lambda_{aa}\Sigma_{ab}\Sigma_{bb}^{-1}$
and hence:
\begin{equation}
\begin{split}
\mu_{a\vert{}b}&=\boldsymbol{\mu}_{a}-\Lambda_{aa}^{-1}\Lambda_{ab}(\mathbf{x}_{b}-\boldsymbol{\mu}_{b})\\
&=\boldsymbol{\mu}_{a}+\Lambda_{aa}^{-1}\Lambda_{aa}\Sigma_{ab}\Sigma_{bb}^{-1}(\mathbf{x}_{b}-\boldsymbol{\mu}_{b})\\
&=\boldsymbol{\mu}_{a}+\Sigma_{ab}\Sigma_{bb}^{-1}(\mathbf{x}_{b}-\boldsymbol{\mu}_{b})
\end{split}
\label{eqn:meanagivenb}
\end{equation}

\subsubsection{Marginal Gaussian distributions}
(Preliminary)
\begin{conclusion}
Suppose $A,B,C, D$ are matrices of dimension
$p\times{}p,p\times{}q,q\times{}p,q\times{}q$ respectively. Besides,
    $D$ is \emph{invertible}. Then we have:
\begin{equation}
\text{det}(\begin{bmatrix}A&B\\C&D\\\end{bmatrix})=\text{det}(D)\text{det}(A-BD^{-1}C)
\label{eqn:blockmatricesdet}
\end{equation}
\end{conclusion}
\begin{proof}
\putansline{6}{6}
\end{proof}
We calculate $\Pr(\mathbf{x}_{a})$ by integrated out $\mathbf{x}_b$:
\begin{equation}
\Pr(\mathbf{x}_a)=\int{}\Pr(\mathbf{x}_{a},\mathbf{x}_{b})\text{d}\mathbf{x}_{b}
\end{equation}


Because we want to integrate out $\mathbf{x}_b$, we should firstly consider
terms that are functional to $\mathbf{x}_b$ in
\eqref{eqn:abquadratic}. Since $b$ is symmetric to $a$ in
\eqref{eqn:abquadratic}, we can directly use
\eqref{eqn:expandexponent}, \eqref{eqn:covagivenb} and
\eqref{eqn:meanagivenb} to derive:
\begin{equation}
\begin{split}
&-\frac{1}{2}\mathbf{x}_{b}^{\mathrm{T}}\Lambda_{bb}\mathbf{x}_{b}+\mathbf{x}_{b}^{\mathrm{T}}\{\Lambda_{bb}\boldsymbol{\mu}_{b}-\Lambda_{ba}(\mathbf{x}_{a}-\boldsymbol{\mu}_{a})\}\\
=&-\frac{1}{2}\mathbf{x}_{b}^{\mathrm{T}}\Lambda_{bb}\mathbf{x}_{b}+\mathbf{x}_{b}^{\mathrm{T}}\mathbf{m}\quad{}(\text{where }\mathbf{m}=\Lambda_{bb}\boldsymbol{\mu}_{b}-\Lambda_{ba}(\mathbf{x}_{a}-\boldsymbol{\mu}_{a}))\\
=&-\frac{1}{2}\mathbf{x}_{b}^{\mathrm{T}}\Lambda_{bb}\mathbf{x}_{b}+\mathbf{x}_{b}^{\mathrm{T}}\Lambda_{bb}(\Lambda_{bb}^{-1}\mathbf{m})-\frac{1}{2}(\Lambda_{bb}^{-1}\mathbf{m})^{\mathrm{T}}\Lambda_{bb}(\Lambda_{bb}^{-1}\mathbf{m})+\frac{1}{2}(\Lambda_{bb}^{-1}\mathbf{m})^{\mathrm{T}}\Lambda_{bb}(\Lambda_{bb}^{-1}\mathbf{m})\\
=&-\frac{1}{2}(\mathbf{x}_{b}-\Lambda_{bb}^{-1}\mathbf{m})^{\mathrm{T}}\Lambda_{bb}(\mathbf{x}_{b}-\Lambda_{bb}^{-1}\mathbf{m})+\frac{1}{2}\mathbf{m}^{\mathrm{T}}\Lambda_{bb}^{-1}\mathbf{m}
\end{split}
\label{eqn:onlybinquadratic}
\end{equation}
Note that $\mathbf{m}$ doesn't contain $\mathbf{x}_b$ and the first
term in the right-hand side of \eqref{eqn:onlybinquadratic} is the
standard quadratic form. Assuming $\mathbf{x}_{a}$ is $M$ dimensional, integration results:
\begin{equation}
\int{}\exp{}\{-\frac{1}{2}(\mathbf{x}_{b}-\Lambda_{bb}^{-1}\mathbf{m})^{\mathrm{T}}\Lambda_{bb}(\mathbf{x}_{b}-\Lambda_{bb}^{-1}\mathbf{m})\}\text{d}\mathbf{x}_{b}=(2\pi)^{\frac{D-M}{2}}\vert{}\Lambda_{bb}^{-1}\vert{}^{\frac{1}{2}}
\label{eqn:intbresult}
\end{equation}
while $\frac{1}{2}\mathbf{m}^{\mathrm{T}}\Lambda_{bb}^{-1}\mathbf{m}$
remains as exponent. We firstly expand it as:
\begin{equation}
\begin{split}
\frac{1}{2}\mathbf{m}^{\mathrm{T}}\Lambda_{bb}^{-1}\mathbf{m}=&\frac{1}{2}(\Lambda_{bb}\boldsymbol{\mu}_{b}-\Lambda_{ba}(\mathbf{x}_{a}-\boldsymbol{\mu}_{a}))^{\mathrm{T}}\Lambda_{bb}^{-1}(\Lambda_{bb}\boldsymbol{\mu}_{b}-\Lambda_{ba}(\mathbf{x}_{a}-\boldsymbol{\mu}_{a}))\\
=&\frac{1}{2}\{\boldsymbol{\mu}_{b}^{\mathrm{T}}\Lambda_{bb}\boldsymbol{\mu}_{b}-\boldsymbol{\mu}_{b}^{\mathrm{T}}\Lambda_{ba}(\mathbf{x}_{a}-\boldsymbol{\mu}_{a})-(\mathbf{x}_{a}-\boldsymbol{\mu}_{a})^{\mathrm{T}}\Lambda_{ab}\boldsymbol{\mu}_{b}+(\mathbf{x}_{a}-\boldsymbol{\mu}_{a})^{\mathrm{T}}\Lambda_{ab}\Lambda_{bb}^{-1}\Lambda_{ba}(\mathbf{x}_{a}-\boldsymbol{\mu}_{a})\}\\
=&\frac{1}{2}(\mathbf{x}_{a}-\boldsymbol{\mu}_{a})^{\mathrm{T}}\Lambda_{ab}\Lambda_{bb}^{-1}\Lambda_{ba}(\mathbf{x}_{a}-\boldsymbol{\mu}_{a})-(\mathbf{x}_{a}-\boldsymbol{\mu}_{a})^{\mathrm{T}}\Lambda_{ab}\boldsymbol{\mu}_{b}+\frac{1}{2}\boldsymbol{\mu}_{b}^{\mathrm{T}}\Lambda_{bb}\boldsymbol{\mu}_{b}
\end{split}
\label{eqn:rmm}
\end{equation}


Then we consider terms that are not functional to $\mathbf{x}_b$ in
\eqref{eqn:abquadratic}:
\begin{equation}
-\frac{1}{2}(\mathbf{x}_{a}-\boldsymbol{\mu}_{a})^{\mathrm{T}}\Lambda_{aa}(\mathbf{x}_{a}-\boldsymbol{\mu}_{a})+(\mathbf{x}_{a}-\boldsymbol{\mu}_{a})^{\mathrm{T}}\Lambda_{ab}\boldsymbol{\mu}_{b}-\frac{1}{2}\boldsymbol{\mu}_{b}^{\mathrm{T}}\Lambda_{bb}\boldsymbol{\mu}_{b}
\label{eqn:hn2dwb}
\end{equation}


Combining \eqref{eqn:rmm} and \eqref{eqn:hn2dwb} we get:
\begin{equation}
\exp{}\{-\frac{1}{2}(\mathbf{x}_{a}-\boldsymbol{\mu}_{a})^{\mathrm{T}}(\Lambda_{aa}-\Lambda_{ab}\Lambda_{bb}^{-1}\Lambda_{ba})(\mathbf{x}_{a}-\boldsymbol{\mu}_{a})\}
\label{eqn:margmd}
\end{equation}
Then we can prove it is normalized by transforming original
normalization term through \eqref{eqn:blockmatricesdet} and substituting
\eqref{eqn:intbresult} into it:
\begin{equation}
\begin{split}
&\frac{1}{(2\pi)^{\frac{D}{2}}}\frac{1}{\vert{}\Sigma\vert{}^{\frac{1}{2}}}(2\pi)^{\frac{D-M}{2}}\vert{}\Lambda_{bb}^{-1}\vert{}^{\frac{1}{2}}\\
=&\frac{1}{(2\pi)^{\frac{M}{2}}}\vert{}\Lambda\vert{}^{\frac{1}{2}}\frac{1}{\vert{}\Lambda_{bb}\vert{}^{\frac{1}{2}}}\quad{}(\because{}AA^{-1}=I,\vert{}A\vert{}\vert{}A^{-1}\vert{}=1)\\
=&\frac{1}{(2\pi)^{\frac{M}{2}}}\frac{(\vert{}\Lambda_{bb}\vert{}\vert{}\Lambda_{aa}-\Lambda_{ab}\Lambda_{bb}^{-1}\Lambda_{ba}\vert{})^{\frac{1}{2}}}{\vert{}\Lambda_{bb}\vert{}^{\frac{1}{2}}}\\
=&\frac{1}{(2\pi)^{\frac{M}{2}}}\frac{1}{\vert{}(\Lambda_{aa}-\Lambda_{ab}\Lambda_{bb}^{-1}\Lambda_{ba})^{-1}\vert{}^{\frac{1}{2}}}\\
\end{split}
\end{equation}
\begin{conclusion}
If two sets of random variables are jointly Gaussian, then the
marginal probability distribution of one set integrated out the other
is again Gaussian.
\begin{equation}
\Pr(\mathbf{x}_a)=\int{}\Pr(\mathbf{x}_{a},\mathbf{x}_{b})\text{d}\mathbf{x}_{b}=\mathcal{N}(\mathbf{x}_{a}\vert{}\boldsymbol{\mu}_{a},\Sigma_{aa})
\label{eqn:marggaussian}
\end{equation}
\end{conclusion}

\subsubsection{Bayes' theorem for Gaussian variables}
Up to now, the whole story is given a quadratic form
$(\mathbf{x}-\boldsymbol{\mu})^{\mathrm{T}}\Lambda(\mathbf{x}-\boldsymbol{\mu})$,
    we could partition $\mathbf{x},\boldsymbol{\mu},\Lambda$ with
    consistent size and transform it into two quadratic forms:
\begin{equation}
-\frac{1}{2}\mathbf{x}^{\mathrm{T}}\Lambda\mathbf{x}=-\frac{1}{2}(\mathbf{x}_{b}-\Lambda_{bb}^{-1}\mathbf{m})^{\mathrm{T}}\Lambda_{bb}(\mathbf{x}_{b}-\Lambda_{bb}^{-1}\mathbf{m})-\frac{1}{2}(\mathbf{x}_{a}-\boldsymbol{\mu}_a)^{\mathrm{T}}(\Lambda_{aa}-\Lambda_{ab}\Lambda_{bb}^{-1}\Lambda_{ba})(\mathbf{x}_{a}-\boldsymbol{\mu}_a)
\label{eqn:quadraticpartition}
\end{equation}
where
$\mathbf{m}=\Lambda_{bb}\boldsymbol{\mu}_{b}-\Lambda_{ba}(\mathbf{x}_{a}-\boldsymbol{\mu}_a)$
and thus ``$\mu$'' for the first term on the right-hand side of
\eqref{eqn:quadraticpartition} is
$\boldsymbol{\mu}_{b}-\Lambda_{bb}^{-1}\Lambda_{ba}(\mathbf{x}_{a}-\boldsymbol{\mu}_a)$
which is a linear function to $\mathbf{x}_a$.


Hence, the inverse operation can be applied to
$\Pr(\mathbf{x})=\mathcal{N}(\mathbf{x}\vert\boldsymbol{\mu},\Lambda^{-1})$,
    $\Pr(\mathbf{y}\vert\mathbf{x})=\mathcal{N}(\mathbf{y}\vert{}A\mathbf{x}+\mathbf{b},L^{-1})$
    resulting their joint Gaussian distribution
    $\Pr(\mathbf{z})=\Pr(\mathbf{x},\mathbf{y})=\Pr(\mathbf{y}\vert{}\mathbf{x})\Pr(\mathbf{x})$.


Suppose the quadratic form is
$(\mathbf{z}-\mathbb{E}[\mathbf{z}])^{\mathrm{T}}R(\mathbf{z}-\mathbb{E}[\mathbf{z}])$
and
$R=\begin{bmatrix}R_{xx}&R_{xy}\\R_{yx}&R_{yy}\\\end{bmatrix},\mathbb{E}[\mathbf{z}]=\begin{bmatrix}\mathbb{E}[\mathbf{z}]_{x}\\\mathbb{E}[\mathbf{z}]_{y}\end{bmatrix}$,
    then we calculate them according to \eqref{eqn:quadraticpartition}:
\begin{itemize}
\item Obviously, $R_{yy}=L$,
    $\mathbb{E}[\mathbf{z}]_{x}=\boldsymbol{\mu}$.
\item
$\because{}A\mathbf{x}+\mathbf{b}=\mathbb{E}[\mathbf{z}]_{y}-R_{yy}^{-1}R_{yx}(\mathbf{x}-\mathbb{E}[\mathbf{z}]_x)\therefore{}A=-R_{yy}^{-1}R_{yx}=-L^{-1}R_{yx}$,
    i.e., $R_{yx}=-LA$.
\item Now
$A\mathbf{x}+\mathbf{b}=\mathbb{E}[\mathbf{z}]_{y}+A(\mathbf{x}-\boldsymbol{\mu})$,
    thus, $\mathbb{E}[\mathbf{z}]_{y}=A\boldsymbol{\mu}+\mathbf{b}$.
\item Since $R$ is required to be symmetric,
    $R_{xy}=R_{yx}^{\mathrm{T}}=-A^{\mathrm{T}}L$.
\item
$\because{}R_{xx}-R_{xy}R_{yy}^{-1}R_{yx}=R_{xx}-A^{\mathrm{T}}LA=\Lambda\therefore{}R_{xx}=\Lambda+A^{\mathrm{T}}LA$.
\end{itemize}
Use \eqref{eqn:schurcomplement}, we know that:
\begin{equation}
\begin{split}
\Pr(\mathbf{z})&=\mathcal{N}(\mathbf{z}\vert{}\begin{bmatrix}\boldsymbol{\mu}\\A\boldsymbol{\mu}+\mathbf{b}\\\end{bmatrix},\begin{bmatrix}\Lambda+A^{\mathrm{T}}LA&-A^{\mathrm{T}}L\\-LA&L\\\end{bmatrix}^{-1})\\
&=\mathcal{N}(\mathbf{z}\vert{}\begin{bmatrix}\boldsymbol{\mu}\\A\boldsymbol{\mu}+\mathbf{b}\\\end{bmatrix},\begin{bmatrix}\Lambda^{-1}&\Lambda^{-1}A^{\mathrm{T}}\\A\Lambda^{-1}&L^{-1}+A\Lambda^{-1}A^{\mathrm{T}}\\\end{bmatrix})
\end{split}    
\label{eqn:gaussianxyjoint}
\end{equation}

\subsubsection{Maximum likelihood for the Gaussian}
(Preliminary)
\begin{definition}
The trace of a square matrix is the sum of the elements on its
diagonal, i.e., $\text{Tr}(A_{n\times{}n})=\sum_{i=1}^{n}A_{ii}$.
\end{definition}
\begin{conclusion}
\begin{equation}
\text{Tr}(AB)=\text{Tr}(BA)
\label{eqn:exchangetr}    
\end{equation}
By which we can generalize to \emph{cyclic} property of the trace
operator: $\text{Tr}(ABC)=\text{Tr}(CAB)=\text{Tr}(BCA)$.
\end{conclusion}
\begin{proof}
Suppose $A,B$ is of size $n\times{}m$ and $m\times{}n$.


$\forall{}i\in{}1,\ldots,n:(AB)_{ii}=\sum_{k=1}^{m}A_{ik}B_{ki}$ Thus
$\text{Tr}(AB)=\sum_{i=1}^{n}\sum_{k=1}^{m}A_{ik}B_{ki}$.


The other, $\text{Tr}(BA)=\sum_{i=1}^{m}\sum_{k=1}^{n}B_{ik}A_{ki}$.
Obviously, in these two cases, each element of $A$ is matched to a corresponding element
of $B$ in the same way.
\end{proof}
\begin{conclusion}
\begin{equation}
\text{Tr}(\mathbf{x}^{\mathrm{T}}A\mathbf{x})=\text{Tr}(A\mathbf{x}\mathbf{x}^{\mathrm{T}})
\label{eqn:trquadraticchange}
\end{equation}
\end{conclusion}
\begin{proof}
Obviously, the left-hand side equals
$\sum_{i=1}^{M}\sum_{j=1}^{M}\mathbf{x}_{i}\mathbf{x}_{j}A_{ij}$.


$\mathbf{x}\mathbf{x}^{\mathrm{T}}=\begin{bmatrix}\mathbf{x}_{1}\mathbf{x}_{1}&\cdots{}&\mathbf{x}_{1}\mathbf{x}_{M}\\\vdots{}&\ddots{}&\vdots{}\\\mathbf{x}_{M}\mathbf{x}_{1}&\cdots{}&\mathbf{x}_{M}\mathbf{x}_{M}\\\end{bmatrix}$
and suppose $A\mathbf{x}\mathbf{x}^{\mathrm{T}}=R$, then
$R_{ii}=\sum_{j=1}^{M}A_{ij}\mathbf{x}_{j}\mathbf{x}_{i}$. Trace is
the sum of elements in diagonal, thus it equals
$\sum_{i=1}^{M}\sum_{j=1}^{M}A_{ij}\mathbf{x}_{j}\mathbf{x}_{i}$.
\end{proof}
\begin{conclusion}
\begin{equation}
\frac{\partial{}\text{Tr}(AB)}{\partial{}x}=\text{Tr}(\frac{\partial{}A}{\partial{}x}B)
\label{eqn:tediousequation}
\end{equation}
when $B$ is fixed or is not functional to $x$.
\end{conclusion}
\begin{proof}
Just write out the subscripts.
\end{proof}
\begin{conclusion}
\begin{equation}
\frac{\partial}{\partial{}x}(A^{-1})=-A^{-1}\frac{\partial{}A}{\partial{}x}A^{-1}
\label{eqn:dxAinv}
\end{equation}
\begin{proof}
$\because{}0=\frac{\partial{}I}{\partial{}x}=\frac{\partial{}AA^{-1}}{\partial{}x}=\frac{\partial{}A}{\partial{}x}A^{-1}+A\frac{\partial{}A^{-1}}{\partial{x}}$
\end{proof}
\end{conclusion}
\begin{conclusion}
For a symmetric, strictly positive definite matrix $A_{M\times{}M}$, we have:
\begin{equation}
\frac{\partial}{\partial{}x}\ln{}\vert{}A\vert{}=\text{Tr}(A^{-1}\frac{\partial{}A}{\partial{}x})
\label{eqn:appc22}
\end{equation}
\end{conclusion}
\begin{proof}
$\because{}A=A^{\mathrm{T}}\therefore{}\text{ by
    \eqref{eqn:detofsymmetric} we have
}\vert{}A\vert{}=\prod_{i=1}^{M}\lambda_{i}$


Then the left-hand side can be expressed as:
\begin{equation}
\sum_{i=1}^{M}\frac{1}{\lambda_i}\frac{\partial{}\lambda_{i}}{\partial{}x}
\label{eqn:leftofc22}
\end{equation}


Use \eqref{eqn:factcovariance} and \eqref{eqn:factprecise}, we
express $A^{-1}\frac{\partial{}A}{\partial{}x}$ as:
\begin{equation}
\begin{split}
A^{-1}\frac{\partial{}A}{\partial{}x}&=\sum_{i=1}^{M}\frac{1}{\lambda_{i}}\mathbf{u}_{i}\mathbf{u}_{i}^{\mathrm{T}}\sum_{j=1}^{M}\frac{\partial{}\lambda_{j}\mathbf{u}_{j}\mathbf{u}_{j}^{\mathrm{T}}}{\partial{}x}\\
&=\sum_{i=1}^{M}\sum_{j=1}^{M}\frac{1}{\lambda_{i}}\mathbf{u}_{i}\mathbf{u}_{i}^{\mathrm{T}}\frac{\partial{}\lambda_{j}\mathbf{u}_{j}\mathbf{u}_{j}^{\mathrm{T}}}{\partial{}x}\\
&=\sum_{i=1}^{M}\sum_{j=1}^{M}\frac{1}{\lambda_{i}}\mathbf{u}_{i}\mathbf{u}_{i}^{\mathrm{T}}(\frac{\partial{}\lambda_{j}}{\partial{}x}\mathbf{u}_{j}\mathbf{u}_{j}^{\mathrm{T}}+\lambda_{j}\frac{\partial{}\mathbf{u}_{j}\mathbf{u}_{j}^{\mathrm{T}}}{\partial{}x})\\
&=\sum_{i=1}^{M}\frac{1}{\lambda_i}\frac{\partial{}\lambda_i}{\partial{}x}\mathbf{u}_{i}\mathbf{u}_{i}^{\mathrm{T}}+\sum_{i=1}^{M}\sum_{j=1}^{M}\frac{\lambda_j}{\lambda_i}\mathbf{u}_{i}\mathbf{u}_{i}^{\mathrm{T}}\frac{\partial{}\mathbf{u}_{j}\mathbf{u}_{j}^{\mathrm{T}}}{\partial{}x}\quad{}(\because{}\mathbf{u}_{i}^{\mathrm{T}}\mathbf{u}_{j}=I_{ij})
\end{split}
\label{eqn:c22expand}
\end{equation}


$\because{}\forall{}i\in{}1,\ldots,M:\text{Tr}(\mathbf{u}_{i}\mathbf{u}_{i}^{\mathrm{T}})=\Vert{}\mathbf{u}_{i}\Vert^{2}=1\therefore{}\text{Tr}(\sum_{i=1}^{M}\frac{1}{\lambda_i}\frac{\partial{}\lambda_i}{\partial{}x}\mathbf{u}_{i}\mathbf{u}_{i}^{\mathrm{T}})=\sum_{i=1}^{M}\frac{1}{\lambda_i}\frac{\partial{}\lambda_i}{\lambda{}x}\text{Tr}(\mathbf{u}_{i}\mathbf{u}_{i}^{\mathrm{T}})=\text{\eqref{eqn:leftofc22}}$.
Hence, next step is to prove the remaining term in the right-hand side
of $\eqref{eqn:c22expand}$ vanishes.


\begin{equation}
\begin{split}
&\text{Tr}\{\sum_{i=1}^{M}\sum_{j=1}^{M}\frac{\lambda_j}{\lambda_i}\mathbf{u}_{i}\mathbf{u}_{i}^{\mathrm{T}}(\frac{\partial{}\mathbf{u}_j}{\partial{}x}\mathbf{u}_{j}^{\mathrm{T}}+\mathbf{u}_{j}\frac{\partial{}\mathbf{u}_{j}^{\mathrm{T}}}{\partial{}x})\}\quad{}(\because{}(AB)'=A'B+AB')\\
&=\text{Tr}\{\sum_{i=1}^{M}\sum_{j=1}^{M}\frac{\lambda_j}{\lambda_i}\mathbf{u}_{i}\mathbf{u}_{i}^{\mathrm{T}}\frac{\partial{}\mathbf{u}_j}{\partial{}x}\mathbf{u}_{j}^{\mathrm{T}}\}+\text{Tr}(\sum_{i=1}^{M}\mathbf{u}_{i}\frac{\partial{}\mathbf{u}_{i}^{\mathrm{T}}}{\partial{}x})\quad{}(\because{}\mathbf{u}_{i}^{\mathrm{T}}\mathbf{u}_{j}=I_{ij})\\
&=\sum_{i=1}^{M}\sum_{j=1}^{M}\frac{\lambda_j}{\lambda_i}\text{Tr}(\mathbf{u}_{i}\mathbf{u}_{i}^{\mathrm{T}}\frac{\partial{}\mathbf{u}_j}{\partial{}x}\mathbf{u}_{j}^{\mathrm{T}})+\text{Tr}(\sum_{i=1}^{M}\mathbf{u}_{i}\frac{\partial{}\mathbf{u}_{i}^{\mathrm{T}}}{\partial{}x})\\
&=\sum_{i=1}^{M}\sum_{j=1}^{M}\frac{\lambda_j}{\lambda_i}\text{Tr}(\mathbf{u}_{j}^{\mathrm{T}}\mathbf{u}_{i}\mathbf{u}_{i}^{\mathrm{T}}\frac{\partial{}\mathbf{u}_j}{\partial{}x})+\text{Tr}(\sum_{i=1}^{M}\mathbf{u}_{i}\frac{\partial{}\mathbf{u}_{i}^{\mathrm{T}}}{\partial{}x})\quad{}(\because{}\text{Tr}(AB)=\text{Tr}(BA))\\
&=\text{Tr}(\sum_{i=1}^{M}\sum_{j=1}^{M}\frac{\lambda_j}{\lambda_i}\mathbf{u}_{j}^{\mathrm{T}}\mathbf{u}_{i}\mathbf{u}_{i}^{\mathrm{T}}\frac{\partial{}\mathbf{u}_j}{\partial{}x})+\text{Tr}(\sum_{i=1}^{M}\mathbf{u}_{i}\frac{\partial{}\mathbf{u}_{i}^{\mathrm{T}}}{\partial{}x})\\
&=\text{Tr}(\sum_{i=1}^{M}\mathbf{u}_{i}^{\mathrm{T}}\frac{\partial{}\mathbf{u}_{i}}{\partial{}x})+\text{Tr}(\sum_{i=1}^{M}\mathbf{u}_{i}\frac{\partial{}\mathbf{u}_{i}^{\mathrm{T}}}{\partial{}x})\\
&=\text{Tr}(\sum_{i=1}^{M}\frac{\partial{}\mathbf{u}_{i}}{\partial{}x}\mathbf{u}_{i}^{\mathrm{T}})+\text{Tr}(\sum_{i=1}^{M}\mathbf{u}_{i}\frac{\partial{}\mathbf{u}_{i}^{\mathrm{T}}}{\partial{}x})\\
&=\text{Tr}(\sum_{i=1}^{M}\frac{\partial{}\mathbf{u}_{i}\mathbf{u}_{i}^{\mathrm{T}}}{\partial{}x})=\text{Tr}(\frac{\partial{}\sum_{i=1}^{M}\mathbf{u}_{i}\mathbf{u}_{i}^{\mathrm{T}}}{\partial{}x})=\text{Tr}(\frac{\partial{}UU^{\mathrm{T}}}{\partial{}x})=\text{Tr}(\frac{\partial{}I}{\partial{}x})=0
\end{split}
\end{equation}
\end{proof}


Given $\mathbf{X}=\{\mathbf{x}_{1},\ldots,\mathbf{x}_{N}\}$ where each
$\mathbf{x}_i$ is drawn \emph{independently} from a $D$-dimensional
Gaussian distribution. We estimate the parameters of the unknown
distribution by maximizing the log likelihood function:
\begin{equation}
\ln{}\Pr(\mathbf{X}\vert{}\mu,\Sigma)=-\frac{ND}{2}\ln(2\pi)-\frac{N}{2}\ln\vert\Sigma\vert-\frac{1}{2}\sum_{i=1}^{N}(\mathbf{x}_{i}-\boldsymbol{\mu})^{\mathrm{T}}\Sigma^{-1}(\mathbf{x}_{i}-\boldsymbol{\mu})
\label{eqn:loglikelihoodofgaussian}
\end{equation}


Firstly, we calculate the derivative of
\eqref{eqn:loglikelihoodofgaussian} w.r.t $\boldsymbol{\mu}$. Suppose
resulting vector is $\mathbf{r}$,
          $\forall{}i\in{}1,\ldots,D:\mathbf{r}_{i}=\frac{\partial{}\ln\Pr(\mathbf{X}\vert\boldsymbol{\mu},\Sigma)}{\partial{}\mathbf{r}_i}$. 
Consider a quadratic form
$\Delta^2=(\mathbf{x}-\boldsymbol{\mu})^{\mathrm{T}}\Sigma^{-1}(\mathbf{x}-\boldsymbol{\mu})=\sum_{i=1}^{D}\sum_{j=1}^{D}(\mathbf{x}_{i}-\boldsymbol{\mu}_i)(\mathbf{x}_{j}-\boldsymbol{\mu}_j)\Sigma^{-1}_{ij}=\sum_{i=1}^{D}\sum_{j=1}^{D}(\mathbf{x}_{i}\mathbf{x}_{j}-\mathbf{x}_{j}\boldsymbol{\mu}_{i}-\mathbf{x}_{i}\boldsymbol{\mu}_{j}+\boldsymbol{\mu}_{i}\boldsymbol{\mu}_{j})\Sigma^{-1}_{ij}$.
\begin{equation}
\begin{split}
\because{}\frac{\partial{}\Delta^2}{\partial{\boldsymbol{\mu}_i}}&=2\sum_{j=1}^{D}(\boldsymbol{\mu}_{j}-\mathbf{x}_{j})\Sigma^{-1}_{ij}\\
\therefore{}\frac{\partial{}\Delta^2}{\partial{\boldsymbol{\mu}}}&=2\Sigma^{-1}(\boldsymbol{\mu}-\mathbf{x})
\end{split}
\label{eqn:dd2mu}
\end{equation}
According to \eqref{eqn:dd2mu}, we know that
$\frac{\partial{}\ln{}\Pr(\mathbf{X}\vert{}\mu,\Sigma)}{\partial{}\boldsymbol{\mu}}=\sum_{i=1}^{N}\Sigma^{-1}(\mathbf{x}_{n}-\boldsymbol{\mu})$
and set it to zero so that:
\begin{equation}
\mu_{\text{ML}}=\frac{1}{N}\sum_{i=1}^{N}\mathbf{x}_n
\label{eqn:mlofgaussianmean}
\end{equation}


Then we fix $\boldsymbol{\mu}$ in \eqref{eqn:loglikelihoodofgaussian}
to $\boldsymbol{\mu}_{\text{ML}}$ and calculate derivative of
\eqref{eqn:loglikelihoodofgaussian} w.r.t. $\Sigma$. Suppose the answer
is matrix $R$ where
$R_{ij}=\frac{\partial{}\ln\Pr(\mathbf{X}\vert\boldsymbol{\mu},\Sigma)}{\partial{}\Sigma_{ij}}$.
Hence, we can consider the derivative of
\eqref{eqn:loglikelihoodofgaussian} w.r.t. a certain $x$:
\begin{equation}
\begin{split}
\frac{\partial{}\ln\Pr(\mathbf{X}\vert\boldsymbol{\mu},\Sigma)}{\partial{}x}=&-\frac{N}{2}\frac{\ln\vert\Sigma\vert}{\partial{}x}-\frac{1}{2}\frac{\partial{}\sum_{n=1}^{N}(\mathbf{x}_{n}-\boldsymbol{\mu})^{\mathrm{T}}\Sigma^{-1}(\mathbf{x}_{n}-\boldsymbol{\mu})}{\partial{}x}\\
=&-\frac{N}{2}\text{Tr}(\Sigma^{-1}\frac{\partial{}\Sigma}{\partial{}x})-\frac{1}{2}\sum_{n=1}^{N}\frac{\partial{}\text{Tr}((\mathbf{x}_{n}-\boldsymbol{\mu})^{\mathrm{T}}\Sigma^{-1}(\mathbf{x}_{n}-\boldsymbol{\mu}))}{\partial{}x}\quad{}(\because{}\text{\eqref{eqn:appc22}})\\
=&-\frac{N}{2}\text{Tr}(\Sigma^{-1}\frac{\partial{}\Sigma}{\partial{}x})-\frac{1}{2}\sum_{n=1}^{N}\frac{\partial{}\text{Tr}(\Sigma^{-1}(\mathbf{x}_{n}-\boldsymbol{\mu})(\mathbf{x}_{n}-\boldsymbol{\mu})^{\mathrm{T}})}{\partial{}x}\quad{}(\because{}\text{\eqref{eqn:trquadraticchange}})\\
=&-\frac{N}{2}\text{Tr}(\Sigma^{-1}\frac{\partial{}\Sigma}{\partial{}x})-\frac{1}{2}\frac{\partial{}\text{Tr}(\Sigma^{-1}\sum_{n=1}^{N}(\mathbf{x}_{n}-\boldsymbol{\mu})(\mathbf{x}_{n}-\boldsymbol{\mu})^{\mathrm{T}})}{\partial{}x}\\
=&-\frac{N}{2}\text{Tr}(\Sigma^{-1}\frac{\partial{}\Sigma}{\partial{}x})-\frac{1}{2}\text{Tr}(\frac{\partial{}\Sigma^{-1}}{\partial{}x}\sum_{n=1}^{N}(\mathbf{x}_{n}-\boldsymbol{\mu})(\mathbf{x}_{n}-\boldsymbol{\mu})^{\mathrm{T}})\quad{}(\because{}\text{\eqref{eqn:tediousequation}})\\
=&-\frac{N}{2}\text{Tr}(\Sigma^{-1}\frac{\partial{}\Sigma}{\partial{}x})-\frac{1}{2}\text{Tr}(-\Sigma^{-1}\frac{\partial\Sigma}{\partial{}x}\Sigma^{-1}\sum_{n=1}^{N}(\mathbf{x}_{n}-\boldsymbol{\mu})(\mathbf{x}_{n}-\boldsymbol{\mu})^{\mathrm{T}})\quad{}(\because{}\text{\eqref{eqn:dxAinv}})\\
=&-\frac{N}{2}\text{Tr}(\frac{\partial{}\Sigma{}}{\partial{}x}\Sigma^{-1}I)+\frac{1}{2}\text{Tr}(\frac{\partial\Sigma}{\partial{}x}\Sigma^{-1}\sum_{n=1}^{N}(\mathbf{x}_{n}-\boldsymbol{\mu})(\mathbf{x}_{n}-\boldsymbol{\mu})^{\mathrm{T}}\Sigma^{-1})\quad{}(\because{}\text{\eqref{eqn:exchangetr}})\\
=&-\frac{N}{2}\text{Tr}(\frac{\partial{}\Sigma}{\partial{}x}\Sigma^{-1}\Sigma\Sigma^{-1})+\frac{1}{2}\text{Tr}(\frac{\partial\Sigma}{\partial{}x}\Sigma^{-1}\sum_{n=1}^{N}(\mathbf{x}_{n}-\boldsymbol{\mu})(\mathbf{x}_{n}-\boldsymbol{\mu})^{\mathrm{T}}\Sigma^{-1})\\
=&\frac{1}{2}\text{Tr}(\frac{\partial{}\Sigma}{\partial{}x}\Sigma^{-1}\{\sum_{n=1}^{N}(\mathbf{x}_{n}-\boldsymbol{\mu})(\mathbf{x}_{n}-\boldsymbol{\mu})^{\mathrm{T}}-N\Sigma\}\Sigma^{-1})
\end{split}
\end{equation}
Thus, we require that
$(\mathbf{x}_{n}-\boldsymbol{\mu})(\mathbf{x}_{n}-\boldsymbol{\mu})^{\mathrm{T}}-N\Sigma=0$,
    i.e.,
\begin{equation}
\Sigma=\frac{1}{N}\sum_{i=1}^{N}(\mathbf{x}_{n}-\boldsymbol{\mu})(\mathbf{x}_{n}-\boldsymbol{\mu})^{\mathrm{T}}
\label{eqn:mlofgaussiancov}
\end{equation}

\subsubsection{Bayesian inference for the Gaussian}
(Preliminary)
\begin{definition}
Gamma distribution is defined as:
\begin{equation}
\text{Gam}(x\vert{}a,b)=\frac{1}{\gamma(a)}b^{a}x^{a-1}\exp{}(-bx)
\end{equation}
where $a,b>0$.
\end{definition}
\begin{conclusion}
(Ex 2.41) The expectation of gamma distribution is $\mathbb{E}[x]=\frac{a}{b}$.
\end{conclusion}
\begin{proof}
\begin{equation}
\begin{split}
\mathbb{E}[x]&=\int_{0}^{\infty}\frac{1}{\gamma(a)}(bx)^{a}\exp{}(-bx)\text{d}x\\
&=\frac{1}{b\gamma(a)}\int_{0}^{\infty}t^{a}\exp{}(-t)\text{d}t\quad{}(\text{let }t=bx)\\
&=\frac{\gamma(a+1)}{b\gamma(a)}=\frac{a}{b}
\end{split}
\end{equation}
\end{proof}
\begin{conclusion}
(Ex 2.42) The variance of gamma distribution is $\text{var}[x]=\frac{a}{b^2}$.
\end{conclusion}
\begin{proof}
\begin{equation}
\begin{split}
\mathbb{E}[x^2]&=\int_{0}^{\infty}\frac{x^2}{\gamma(a)}b^{a}x^{a-1}\exp{}(-bx)\text{d}x\\
&=\frac{1}{b\gamma(a)}\int_{0}^{\infty}(bx)^{a+1}\exp{}(-bx)\text{d}x\\
&=\frac{1}{b^2\gamma(a)}\int_{0}^{\infty}t^{a+1}\exp{}(-t)\text{d}t\quad{}(\text{let }t=bx)\\
&=\frac{\gamma(a+2)}{b^2\gamma(a)}=\frac{a(a+1)}{b^2}
\end{split}
\end{equation}
\begin{equation}
\text{var}[x]=\mathbb{E}[x^2]-\mathbb{E}[x]^2=\frac{a(a+1)}{b^2}-\frac{a^2}{b^2}=\frac{a}{b^2}
\end{equation}
\end{proof}
\begin{definition}
Wishart distribution for $\Lambda_{D\times{}D}$ is defined as:
\begin{equation}
\mathcal{W}(\Lambda\vert\mathbf{W},v)=B\vert\Lambda\vert^{\frac{v-D-1}{2}}\exp{}(-\frac{1}{2}\text{Tr}(\mathbf{W}^{-1}\Lambda))
\end{equation}
\end{definition}
Suppose the precision matrix $\Lambda$ is known, to 
calculate its posterior $\Pr(\boldsymbol{\mu}\vert\mathbf{X})\varpropto\Pr(\mathbf{X}\vert\boldsymbol{\mu})\Pr(\boldsymbol{\mu})$, we choose the prior to be also Gaussian $\mathcal{N}(\boldsymbol{\mu}\vert\boldsymbol{\mu}_{0},\Lambda_{0}^{-1})$ so that 
the terms in exponent becomes $-\frac{1}{2}\{\boldsymbol{\mu}^{\mathrm{T}}\Lambda_{0}\boldsymbol{\mu}+\sum_{i=1}^{N}(\boldsymbol{\mu}-\mathbf{x}_{i})^{\mathrm{T}}\Lambda{}(\boldsymbol{\mu}-\mathbf{x}_{i}\}$.
We can complete the square according to \eqref{eqn:expandexponent}.  The second order term of $\boldsymbol{\mu}$ is $-\frac{1}{2}\boldsymbol{\mu}^{\mathrm{T}}(\Lambda_{0}+N\Lambda)\boldsymbol{\mu}$. Thus the precision matrix $\Lambda_{N}$ of $\Pr(\boldsymbol{\mu}\vert\mathbf{X})$ is:
\begin{equation}
\Lambda_{N}=\Lambda_{0}+N\Lambda
\end{equation}
Then we check the first order term of $\boldsymbol{\mu}$ is $\boldsymbol{\mu}^{\mathrm{T}}(\Lambda_{0}\boldsymbol{\mu}_{0}+\Lambda{}\sum_{i=1}^{N}\mathbf{x}_{i})$. 
The mean $\boldsymbol{\mu}_{N}$ of $\Pr(\boldsymbol{\mu}\vert\mathbf{X})$ is:
\begin{equation}
\begin{split}
\boldsymbol{\mu}_{N}&=(\Lambda_{0}+N\Lambda)^{-1}(\Lambda_{0}\boldsymbol{\mu}_{0}+\Lambda{}\sum_{i=1}^{N}\mathbf{x}_{i})\\
&=(\Lambda_{0}+N\Lambda)^{-1}(\Lambda_{0}\boldsymbol{\mu}_{0}+N\Lambda{}\boldsymbol{\mu}_{\text{ML}})
\end{split}
\end{equation}


Suppose the mean $\mu$ is known, to  calculate its posterior $\Pr(\lambda\vert\mathbf{X})\varpropto\Pr(\mathbf{X}\vert\lambda)\Pr(\lambda)$, we firstly check the likelihood function:
\begin{equation}
\Pr(\mathbf{X}\vert\lambda)\varpropto\lambda^{\frac{N}{2}}\exp{}\{-\frac{\lambda}{2}\sum_{i=1}^{N}(x_{i}-\mu)^2\}
\end{equation}
If we choose the prior to be gamma distribution $\text{Gam}(\lambda\vert{}a_0,b_0)$, then the posterior keeps the same form $\text{Gam}(\lambda\vert{}a_{N},b_{N})$ with the prior:
\begin{equation}
\Pr(\lambda\vert\mathbf{X})\varpropto{}\lambda^{a_{0}-1}\lambda^{\frac{N}{2}}\exp{}\{-b_{0}\lambda-\frac{\lambda}{2}\sum_{i=1}^{N}(x_{i}-\mu)^2\}
\end{equation}
where we have:
\begin{gather}
a_{N}=a_{0}+\frac{N}{2}\\
b_{N}=b_{0}+\frac{N}{2}\sigma^{2}_{\text{ML}}
\end{gather}
For multivariate situation, the likelihood function is:
\begin{equation}
\Pr(\mathbf{X}\vert\Lambda)\varpropto\vert\Lambda\vert^{\frac{N}{2}}\exp{}(-\frac{1}{2}\sum_{i=1}^{N}(\mathbf{x}_{i}-\boldsymbol{\mu})^{\mathrm{T}}\Lambda(\mathbf{x}_{i}-\boldsymbol{\mu}))
\end{equation}
If we choose the prior to be Wishart distribution $\mathcal{W}(\Lambda\vert\mathbf{W}_{0},v_{0})$, then the posterior keeps the same form $\mathcal{W}(\Lambda\vert\mathbf{W}_{N},v_{N})$ with the prior (Ex 2.45):
\begin{equation}
\begin{split}
\Pr(\Lambda\vert\mathbf{X})&\varpropto{}\vert\Lambda\vert^{\frac{v_{0}-D-1}{2}}\vert\Lambda\vert^{\frac{N}{2}}\exp{}\{-\frac{1}{2}(\text{Tr}(\mathbf{W}_{0}^{-1}\Lambda)+\sum_{i=1}^{N}(\mathbf{x}_{i}-\boldsymbol{\mu})^{\mathrm{T}}\Lambda(\mathbf{x}_{i}-\boldsymbol{\mu}))\}\\
&=\vert\Lambda\vert^{\frac{N+v_{0}-D-1}{2}}\exp{}\{-\frac{1}{2}(\text{Tr}(\mathbf{W}_{0}^{-1}\Lambda)+\text{Tr}(\sum_{i=1}^{N}(\mathbf{x}_{i}-\boldsymbol{\mu})^{\mathrm{T}}\Lambda(\mathbf{x}_{i}-\boldsymbol{\mu})))\}\\
&=\vert\Lambda\vert^{\frac{N+v_{0}-D-1}{2}}\exp{}\{-\frac{1}{2}(\text{Tr}(\mathbf{W}_{0}^{-1}\Lambda)+\text{Tr}(\sum_{i=1}^{N}\Lambda(\mathbf{x}_{i}-\boldsymbol{\mu})(\mathbf{x}_{i}-\boldsymbol{\mu})^{\mathrm{T}}))\}\quad{}(\because\text{\eqref{eqn:trquadraticchange}})\\
&=\vert\Lambda\vert^{\frac{N+v_{0}-D-1}{2}}\exp{}\{-\frac{1}{2}(\text{Tr}(\mathbf{W}_{0}^{-1}\Lambda)+\text{Tr}(\sum_{i=1}^{N}(\mathbf{x}_{i}-\boldsymbol{\mu})(\mathbf{x}_{i}-\boldsymbol{\mu})^{\mathrm{T}}\Lambda))\}\quad{}(\because\text{\eqref{eqn:exchangetr}})\\
&=\vert\Lambda\vert^{\frac{N+v_{0}-D-1}{2}}\exp{}\{-\frac{1}{2}\text{Tr}\{(\mathbf{W}_{0}^{-1}+\sum_{i=1}^{N}(\mathbf{x}_{i}-\boldsymbol{\mu})(\mathbf{x}_{i}-\boldsymbol{\mu})^{\mathrm{T}})\Lambda\}\}\\
\end{split}
\end{equation}
where we have:
\begin{gather}
v_{N}=N+v_{0}\\
\mathbf{W}_{N}=(\mathbf{W}_{0}+\sum_{i=1}^{N}(\mathbf{x}_{i}-\boldsymbol{\mu})(\mathbf{x}_{i}-\boldsymbol{\mu})^{\mathrm{T}})^{-1}
\end{gather}


%Suppose both mean $\mu$ and variance $\sigma=\frac{1}{\lambda}$ are unknown, to calculate its posterior $\Pr(\mu,\lambda\vert\mathbf{X})$ we firstly check the form of likelihood function:
%\begin{equation}
%\begin{split}
%\Pr(\mathbf{X}\vert\mu\lambda)&=\prod_{i=1}^{N}(\frac{\lambda}{2\pi})^{\frac{1}{2}}\exp{}\{-\frac{\lambda}{2}(x_{i}-\mu)^2\}\\
%&\varpropto[\lambda^{\frac{1}{2}}\exp{}(-\frac{\lambda\mu^2}{2})]^{N}\exp{}\{-\frac{\lambda}{2}\sum_{i=1}^{N}x_{i}^{2}+\lambda\mu\sum_{i=1}^{N}x_{i}\}
%\end{split}
%\end{equation}

